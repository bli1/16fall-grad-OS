%%%%%%%%%%%%%%%%%%%%%%%%%%%%%%%%%%%%%%%%%
% University/School Laboratory Report
% LaTeX Template
% Version 3.1 (25/3/14)
%
% This template has been downloaded from:
% http://www.Laemphlates.com
%
% Original author:
% Linux and Unix Users Group at Virginia Tech Wiki 
% (https://vtluug.org/wiki/Example_LaTeX_chem_lab_report)
%
% License:
% CC BY-NC-SA 3.0 (http://creativecommons.org/licenses/by-nc-sa/3.0/)
%
%%%%%%%%%%%%%%%%%%%%%%%%%%%%%%%%%%%%%%%%%

%----------------------------------------------------------------------------------------
%	PACKAGES AND DOCUMENT CONFIGURATIONS
%----------------------------------------------------------------------------------------

\documentclass{article}

\usepackage[margin=1.7in]{geometry}
\usepackage{siunitx} % Provides the \SI{}{} and \si{} command for typesetting SI units
\usepackage{graphicx} % Required for the inclusion of images
%\usepackage{natbib} % Required to change bibliography style to APA
\usepackage{amsmath} % Required for some math elements 
\usepackage{float}
\usepackage{hyperref}
\usepackage{algorithm, algpseudocode}% http://ctan.org/pkg/algorithms
\usepackage{algpseudocode}
\setlength\parindent{0pt} % Removes all indentation from paragraphs

%\usepackage{times} % Uncomment to use the Times New Roman font

%----------------------------------------------------------------------------------------
%	DOCUMENT INFORMATION
%----------------------------------------------------------------------------------------

\title{Case Study: Linux Container Runtime Benchmarks \\ Graduate Operating System \\ CSE 60641} % Title

%\author{John \textsc{Smith}} % Author name

\date{\today} % Date for the report
\author{Boyang Li, Bingyu Shen, Chao Zheng}

\begin{document}

\maketitle % Insert the title, author and date

\begin{center}
\begin{tabular}{l r}
Due: & Sep 22, 2016\\ 
\end{tabular}
\end{center}

\section{References Description}

In this section, we briefly describ the main idea of each paper, the detail information of each
paper can be found in the \textbf{Refernece} section.

\subsection{Paper from 1980 and before}

\textbf{Journal Article} \emph{Survey of virtual machine research}\cite{goldberg1974survey} One
of the best and earliest paper discuess the Virtual Machine. There is no reason to ignore this paper  
In this paper, author describe the principles, implementation details and practices of the first
generation of virtual machine that is ready to be put into industry field. In order to realize the
base idea of virtual machine, this paper is worth to study. 

\subsection{Paper from 1990}

\subsection{Paper from 2000}

\textbf{Journal Article} \emph{Virtual machine monitors: Current technology and future trends}
\cite{rosenblum2005virtual}. This is a survey paper introduces the history and details of virtual 
machine(VM) monitor. It describes the motivation of developing VM monitor and implementation 
issues, like CPU virtualization, memory virtualization, I/O virtualization and security issue. The
author also talks about the future trend of the VM monitor. VM as a hardware level virtualization
technology share various ideas with Container, the OS level virtualization technology is worth 
digging through. This paper would be helpful for me to understand the implementation details of VM 
monitor and how do they influence performance.

\medskip

\textbf{Conference Article} \emph{Solaris Zones: Operating System Support for Consolidating 
Commercial Workloads}\cite{price2004solaris}. This paper introduce the Solaris Zones, the 
virtualization technology developed by Sun Microsystems, Inc. As one of the inchoate OS 
level virtualization runtime, Solaris Zones and Current container runtimes share many common 
ideas, like namespace isolation. This paper is important for understanding how container 
work internally by applying namespace isolation facilities.

\medskip

\textbf{Conference Article} \emph{Quantifying the performance isolation properties of virtualization 
systems}\cite{matthews2007quantifying}.

\medskip

\textbf{Conference Article} \emph{kvm: the Linux virtual machine monitor}\cite{kivity2007kvm}. 
In this paper, the author intorduced the Kernel-based Virtual Machine (kvm), which take advantages of
x86 virtualization extension to implement a virtual machine monitor for Linux system. This 
paper show the general ideas of how to virtualize MMU, I/O, CPU and memory on the Linux system.
Since we are planning to running the benchmark on Unix like system, reading this paper can help us 
understand what cause the overhead of running VM.

\subsection{Paper from 2010}

\textbf{Journal Article} \emph{Updated Performance Comparison of Virtual Machines and Linux 
Containers}\cite{felter2014updated}. In this paper, the author compare the performance of 
native, container(i.e. Docker) and virtual machine environments using recent hardware and common 
used software. We are planning to conduct same kind of benchmarks on one more configuration
(i.e. host OS with namespace isolation). This paper is worth to read and several benchmarks 
design principles are worth to refer.

\medskip

\textbf{Journal Article} \emph{Analysis of docker security}\cite{bui2015analysis}. This paper 
discuss the potential security issue of the Docker container runtime. It also introduces the 
isolation facilities of Docker. We now know that compare to virtual machine, container runtime 
has lower performance overheads. But does that demands at the cost of security? We will return to
read this paper and figure out the solution. 

\medskip

\textbf{Conference Article} \emph{Performance evaluation of container-based virtualization for 
high performance computing environments}\cite{xavier2013performance}

\medskip

\textbf{Conference Article} \emph{Power Consumption of Virtualization Technologies: an 
Empirical Investigation}\cite{morabito2015power}

\medskip

\textbf{Webpage} \emph{CoreOS is building a container runtime, Rocket}\cite{rocketwb}

\medskip

\textbf{Webpage} \emph{Docker project official website}\cite{dockerwb}

\medskip

\textbf{Webpage} \emph{Project Atomic}\cite{RHELatomic}

%----------------------------------------------------------------------------------------
%	BIBLIOGRAPHY
%----------------------------------------------------------------------------------------
% print out all references without citing

\bibliographystyle{abbrv}

\bibliography{sample}

%----------------------------------------------------------------------------------------

\end{document}
