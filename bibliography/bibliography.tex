%%%%%%%%%%%%%%%%%%%%%%%%%%%%%%%%%%%%%%%%%
% University/School Laboratory Report
% LaTeX Template
% Version 3.1 (25/3/14)
%
% This template has been downloaded from:
% http://www.Laemphlates.com
%
% Original author:
% Linux and Unix Users Group at Virginia Tech Wiki 
% (https://vtluug.org/wiki/Example_LaTeX_chem_lab_report)
%
% License:
% CC BY-NC-SA 3.0 (http://creativecommons.org/licenses/by-nc-sa/3.0/)
%
%%%%%%%%%%%%%%%%%%%%%%%%%%%%%%%%%%%%%%%%%

%----------------------------------------------------------------------------------------
%	PACKAGES AND DOCUMENT CONFIGURATIONS
%----------------------------------------------------------------------------------------

\documentclass{article}

\usepackage[margin=1.7in]{geometry}
\usepackage{siunitx} % Provides the \SI{}{} and \si{} command for typesetting SI units
\usepackage{graphicx} % Required for the inclusion of images
%\usepackage{natbib} % Required to change bibliography style to APA
\usepackage{amsmath} % Required for some math elements 
\usepackage{float}
\usepackage{hyperref}
\usepackage{algorithm, algpseudocode}% http://ctan.org/pkg/algorithms
\usepackage{algpseudocode}
\setlength\parindent{0pt} % Removes all indentation from paragraphs

%\usepackage{times} % Uncomment to use the Times New Roman font

%----------------------------------------------------------------------------------------
%	DOCUMENT INFORMATION
%----------------------------------------------------------------------------------------

\title{{\textbf{Virtual Machine vs. Container Runtimes: Performance Comparison}} \\
       \vspace{3\baselineskip}
       {\large Bibliography List} \\
       \vspace{3\baselineskip}
       {\large Graduate Operating System} \\ 
       {\large CSE 60641} % Title
      }
%\author{John \textsc{Smith}} % Author name

\date{\today} % Date for the report
\author{Chao Zheng}

\begin{document}

\maketitle % Insert the title, author and date

\begin{center}
\begin{tabular}{l r}
Due:& September 15, 2016\\ 
\end{tabular}
\end{center}

\pagebreak

\section{References Description}

In this section, we briefly describe the main idea of each paper, the detail information of each
paper can be found in the \textbf{Reference} section.

\subsection{Paper from 1980 and before}

\textbf{Journal Article} \emph{Survey of virtual machine research}\cite{goldberg1974survey} One
of the best and earliest paper discuss the Virtual Machine. There is no reason to ignore this paper  
In this paper, author describe the principles, implementation details and practices of the first
generation of virtual machine that is ready to be put into industry field. In order to realize the
base idea of virtual machine, this paper is worth to study. 

\medskip

\textbf{Journal Article} \emph{Virtual storage and virtual machine concepts}
\cite{parmelee1972virtual}. This is the one of the earliest paper about virtual machine 
and virtual storage system. In order to know what cause the overheads of virtual machine, we should
understand the internal infrastructure of virtual machine and the implementation details. This paper 
is worth reading.

\subsection{Paper from 1990}

\textbf{Conference Article} \emph{Using Microbenchmarks To Evaluate System Performance}
\cite{bershad1992using}. This is a brief but very interesting paper. The author point out two 
assumptions underlie the use of microbenchmarks.
\begin{itemize}
    \item The execution time of code path for microbenchmark and real programs are same or at least 
        similar.
    \item The microbenchmarks are somehow represent something important to the overall 
        system performance.
\end{itemize}
The author discusses cache and write buffer when considering I/O performance to show he vulnerability
of the first assumption and identify weaknesses in the second assumption.

\medskip

\textbf{Conference Article} \emph{Exokernel: An operating system architecture for application-level 
resource management}\cite{engler1995exokernel}. This paper introduces the Exokernel system, which 
expose the hardware level resource management library to the application level. It enable the 
application to run with customized resource configuration. The idea of this system is quite 
creative and bold. Will this cause insecure issue of the system? Will the resource be over 
subscribed? How can an end user know how much resources each application require? The high 
citation of this paper caught my eye.

\medskip

\textbf{Conference Article} \emph{Hypervisor-based fault tolerance}\cite{bressoud1995hypervisor}.
This paper presents the idea of implementing a fault-tolerance system that has a replica system 
managed by virtual machine hypervisor. When the main system is down the replica can be used to 
replace it. The idea presented in the paper is another use of virtual machine other than 
resource sharing.

\medskip 

\textbf{Journal Article} \emph{Disco: Running commodity operating systems on scalable 
multiprocessors}\cite{bugnion1997disco}. This is one of the early paper that introduce the 
idea of run multiple commodity operating system on a scalable multiprocessor. This paper is 
important for understanding how to virtualize hardware and running multiple OS on a scalable 
multiprocessor machine.

\subsection{Paper from 2000}

\textbf{Conference Article} \emph{When virtual is better than real}\cite{chen2001virtual}. This is
a position paper. Author argues that most applications and OS currently running on a real machine 
should relocate into a virtual machine. Consider about the overheads of traditional virtual machine,
it is impossible. But with the arise of container technology, some OS, like CoreOS, is running 
every application inside a private container. It is interesting to know when did people first think 
about running applications on highly virtual machine.

\medskip

\textbf{Conference Article} \emph{The design and implementation of Zap: A system for migrating 
computing environments}\cite{osman2002design}. The author introduce the pod concepts, which 
consists bunch of processes and a virtualizaion layer on top of the operating system. The purpose
of processes pod is for processes migrating around different operating systems. The author also 
show that Zap prototype running with Apache web server and X windows desktop application has 
subsecond checkpoint and restart latencies.

\medskip

\textbf{Conference Article} \emph{Terra: A virtual machine-based platform for trusted 
computing}\cite{garfinkel2003terra}. This paper present a flexible platform, which use virtual 
machine monitor that partition a hardware platform into multiple virtual machine for various
applications require heterogeneous environment. This system can be treated as the prototype of
Amazon EC2 and Google App Engine. It is worth to know what kind of role virtual machine play in 
this system and can container runtime replace the VM.

\medskip

\textbf{Conference Article} \emph{Solaris Zones: Operating System Support for Consolidating 
Commercial Workloads}\cite{price2004solaris}. This paper introduce the Solaris Zones, the 
virtualization technology developed by Sun Microsystems, Inc. As one of the inchoate OS 
level virtualization runtime, Solaris Zones and Current container runtimes share many common 
ideas, like namespace isolation. This paper is important for understanding how container 
work internally by applying namespace isolation facilities.

\medskip

\textbf{Conference Article} \emph{Live Migration of Vritual Machines}\cite{clark2005live}. This is
one of the most famous paper in Virtual Machines field. In this paper, the author consider the design 
options of live migration of virtual machine instance on the cluster data center and cluster 
environment. So far there is no report about live migration of container instances, thus it would be 
helpful to understand the purpose and implementation details of virtual machines migration. 

\medskip

\textbf{Journal Article} \emph{Virtual machine monitors: Current technology and future trends}
\cite{rosenblum2005virtual}. This is a survey paper introduces the history and details of virtual 
machine(VM) monitor. It describes the motivation of developing VM monitor and implementation 
issues, like CPU virtualization, memory virtualization, I/O virtualization and security issue. The
author also talks about the future trend of the VM monitor. VM as a hardware level virtualization
technology share various ideas with Container, the OS level virtualization technology is worth 
digging through. This paper would be helpful for me to understand the implementation details of VM 
monitor and how do they influence performance.

\medskip

\textbf{Conference Article} \emph{Virtualization for high-performance computing}
\cite{mergen2006virtualization}. This paper discuss the specific demands of HPC environment which 
often mismatch the solutions provided by legacy operating systems. The author focused on the 
hardware virtualization with emphasis on HPC environments. Reading this paper is helpful for 
knowing what specific demands HPC ask for virtualization technology.

\medskip

\textbf{Conference Article} \emph{Container-based Operating System Virtualization
: a Scalable, High-performance Alternative to Hypervisors}\cite{soltesz2007container}. This paper 
discuss the possibility of replace Hypervisors with Container-based OS virtualization specially 
for HPC environment. Author explores the features of Linux-VServer with current generations of Xen. 
The conclusion is that Linux-VServer provide comparable support for isolation and better performance
than traditional Hypervisors.

\medskip

\textbf{Conference Article} \emph{Quantifying the performance isolation properties of virtualization 
systems}\cite{matthews2007quantifying}. This paper present a special kind of benchmark, the 
performance isolation properties benchmark, which quantifies the degree to which a virtualization
system limits the impact of a misbehaving virtual machine on other well-behaving virtual machines
running on the same physical machine. The experiment results show that hardware level virtualization
technology offer complete isolation in all cases, while the results of OS level virtualization 
technology are varied. Based on this observation, we know that when pursuit low performance overhead 
one may suffer low isolation performance as well. 

\medskip

\textbf{Conference Article} \emph{kvm: the Linux virtual machine monitor}\cite{kivity2007kvm}. 
In this paper, the author introduced the Kernel-based Virtual Machine (kvm), which take advantages of
x86 virtualization extension to implement a virtual machine monitor for Linux system. This 
paper show the general ideas of how to virtualize MMU, I/O, CPU and memory on the Linux system.
Since we are planning to running the benchmark on Unix like system, reading this paper can help us 
understand what cause the overhead of running VM.

\subsection{Paper from 2010}

\textbf{Journal Article} \emph{An Updated Performance Comparison of Virtual Machines and Linux 
Containers}\cite{felter2014updated}. In this paper, the author compare the performance of 
native, container(i.e. Docker) and virtual machine environments using recent hardware and common 
used software. We are planning to conduct same kind of benchmarks on one more configuration
(i.e. host OS with namespace isolation). This paper is worth to read and several benchmarks 
design principles are worth to refer.

\medskip

\textbf{Conference Article} \emph{Virtualization vs Containerization To Support PaaS}
\cite{dua2014virtualization}. In this paper, author explore various container runtime 
implementations and try to figure out which one can fit the PaaS best. One of the major use
of container is offering isolated environment for applications running on cloud. By reading 
this paper, we can understand which performance metric is the important for applying 
container runtime into cloud environment.

\medskip

\textbf{Journal Article} \emph{Analysis of docker security}\cite{bui2015analysis}. This paper 
discuss the potential security issue of the Docker container runtime. It also introduces the 
isolation facilities of Docker. We now know that compare to virtual machine, container runtime 
has lower performance overheads. But does that demands at the cost of security? We will return to
read this paper and figure out the solution. 

\medskip

\textbf{Conference Article} \emph{Performance evaluation of container-based virtualization for 
high performance computing environments}\cite{xavier2013performance}. One of the primary use of
container technology is to deliver isolation environment for application running on heterogeneous 
cluster. The author of this paper conducting benchmarks on large HPC workflows with various 
container runtime. This paper is worth reading because it consider the overhead for large 
cloud applications, which does not show up when running small local benchmark.

\medskip

\textbf{Conference Article} \emph{Power Consumption of Virtualization Technologies: an 
Empirical Investigation}\cite{morabito2015power}. We would not benchmark the power consumption 
of virtualization technologies in our project, but the power consumption is a important 
aspect when lunching application required tens and hundreds of computing nodes. Thus the 
experiment results of this paper can be a good reference when consider about large distributed
application, which we will briefly discuss in our report.

\medskip

\textbf{Conference Article} \emph{Hypervisors vs. Lightweight Virtualization: a Performance 
Comparison}\cite{morabito2015hypervisors}. Another performance benchmark paper. Different than 
the Felter's paper, author of this paper compare KVM with Docker and LXC. We can compare the 
results of our benchmarks with the experiment results of this paper.

\medskip

\textbf{Webpage} \emph{Docker project official website}\cite{dockerwb}. This is the official 
Docker website. You can find almost every thing you need to know for using Docker, from 
general idea to implementation details.  There are several reasons for us to choose Docker,
\begin{itemize}
    \item Docker as one of the most popular container runtime, it is open source, well supported 
        and well documented.
    \item The public Docker registry is used by many users, thus we have wide range of 
        container images to choose.
    \item It is will developed, which means we can focus on the general ideas of container technology
        instead of implementation details.
\end{itemize}
This website should be treated as one of the main reference for conducting experiments on Docker 
container.

\medskip

\textbf{Webpage} \emph{CoreOS is building a container runtime, Rocket}\cite{rocketwb}. Rocket is
a new container runtime developed by people from CoreOS. Since Docker have been adopted in many 
fields, like cloud servers launching, building systems for clustering, it includes a wide range
of functions like building images, running images, uploading and even overlay networking. It has 
become heave-weight and more like a platform instead of a container runtime. Not like Docker 
container, Rocket focus on features like composable, security, easily distribution and open. Since 
we are going to benchmark the performance of Docker, it is worth reading this website to understand
how Rocket different from docker and how Rocket avoid the overheads.

\medskip

\textbf{Webpage} \emph{Singularity: Extreme Mobility of Compute}\cite{singularity}. One of the 
limitations of Docker runtime is that only privileged users have the right to launch a native
container environment. Singularity breaks this limitation and focus on features like portability 
and reproducibility. And the inventor claim that Singularity can fit HPC workflow better 
than other container runtimes. We plan to benchmark the performance of Singularity and compare 
it with Docker. It is worth to know if Singularity imposes any overheads with the capability to 
launch container without privilege.  


%----------------------------------------------------------------------------------------
%	BIBLIOGRAPHY
%----------------------------------------------------------------------------------------
% print out all references without citing

\bibliographystyle{abbrv}

\bibliography{sample}

%----------------------------------------------------------------------------------------

\end{document}
