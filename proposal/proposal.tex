%%%%%%%%%%%%%%%%%%%%%%%%%%%%%%%%%%%%%%%%%
% University/School Laboratory Report
% LaTeX Template
% Version 3.1 (25/3/14)
%
% This template has been downloaded from:
% http://www.Laemphlates.com
%
% Original author:
% Linux and Unix Users Group at Virginia Tech Wiki 
% (https://vtluug.org/wiki/Example_LaTeX_chem_lab_report)
%
% License:
% CC BY-NC-SA 3.0 (http://creativecommons.org/licenses/by-nc-sa/3.0/)
%
%%%%%%%%%%%%%%%%%%%%%%%%%%%%%%%%%%%%%%%%%

%----------------------------------------------------------------------------------------
%	PACKAGES AND DOCUMENT CONFIGURATIONS
%----------------------------------------------------------------------------------------

\documentclass{article}

\usepackage[margin=1.7in]{geometry}
\usepackage{siunitx} % Provides the \SI{}{} and \si{} command for typesetting SI units
\usepackage{graphicx} % Required for the inclusion of images
%\usepackage{natbib} % Required to change bibliography style to APA
\usepackage{amsmath} % Required for some math elements 
\usepackage{float}
\usepackage{hyperref}
\usepackage{algorithm, algpseudocode}% http://ctan.org/pkg/algorithms
\usepackage{algpseudocode}
\setlength\parindent{0pt} % Removes all indentation from paragraphs

%\usepackage{times} % Uncomment to use the Times New Roman font

%----------------------------------------------------------------------------------------
%	DOCUMENT INFORMATION
%----------------------------------------------------------------------------------------

\title{{\textbf{Virtual Machine vs. Container Runtimes: Performance Comparison}} \\
       \vspace{3\baselineskip}
       {\large Project proposal} \\
       \vspace{3\baselineskip}
       {\large Graduate Operating System} \\ 
       {\large CSE 60641} % Title
      }
%\author{John \textsc{Smith}} % Author name

\date{\today} % Date for the report
\author{Boyang Li, Bingyu Shen, Chao Zheng}

\begin{document}

\maketitle % Insert the title, author and date

\begin{center}
\begin{tabular}{l r}
Due:& September 22, 2016\\ 
\end{tabular}
\end{center}
\nocite{*}

\pagebreak

\section{Abstract}

Traditional virtualization machines that use hypervisors to virtualize hardware devices are 
enable heterogenous operating systems(OS) share limited computing resources. Meanwhile, sine each 
operating system need to be started from scratch, it imposes nonnegligible overheads. Recently, 
with the rise of OS level virtualization, it becomes possible to run various OS on a host system 
with very low overheads. In this paper, we compare various aspects of system performance between 
origin namespaces isolation, light weight container runtimes and virtual machines. We also run several
HPC workflows on Notre Dame disc cluster with Docker container runtime\cite{dockerwb}, 
Makeflow\cite{albrecht2012makeflow} and Condor\cite{thain2003condor} to check the overheads of 
light weight container on cluster level.  

\section{Introduction}

\subsection{Linux Namespaces}

\subsection{Singularity}

Singularity\cite{singularity}

\subsection{Docker}

Docker\cite{dockerwb}

\subsection{Virtual Machine}

\subsection{Virtualization on HPC}

\section{Related Work}

\section{Proposed Work}

\subsection{Microbenchmarks}

\subsubsection{System benchmark tools}

\subsection{Macrobenchmarks}

\section{Evaluation}

\section{Timeline}

%----------------------------------------------------------------------------------------
%	BIBLIOGRAPHY
%----------------------------------------------------------------------------------------
% print out all references without citing

\bibliographystyle{abbrv}

\bibliography{sample}

%----------------------------------------------------------------------------------------

\end{document}
